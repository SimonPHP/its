\documentclass[12pt,a4paper,oneside,ngerman]{article}
\usepackage[utf8]{inputenc}
\usepackage{color}
\usepackage{tikz}
\usepackage{amsmath}
\usepackage{amssymb}
\usepackage{calc}
\usepackage{mathtools}
\usepackage{float}
\usepackage{struktex}
\usepackage{ulem}

\title{EZS}
\author{Simon Krücken}

\newcommand\tab[1][1cm]{\hspace*{#1}}

\begin{document}
    
\begin{titlepage}
%    \maketitle
\end{titlepage}
\tableofcontents
\pagebreak

\section{Unterlagenfreier Teil}

1. Wo mit Beschäftigt sich IT-Sicherheit(Definition)?\\
IT-Sicherheit beschäftigt sich mit der Vorbeugung, dem Erkennen und der Reaktion auf Ereignisse, die die Integrität der Daten, die Nutzbarkeit der Systeme und die (digitale) Privatsphäre gefährden.\\

2. Vor welchen drei Gefährdungen müssen Rechner- und Netzwehrkomponenten geschützt werden?\\
\begin{itemize}
	\item Spionage
	\item Sabotage
	\item Missbrauch
\end{itemize}

3. Nennen Sie die in der Vorlesung genannten drei Schutzziele für Rechner- und Netzkomponenten.\\
\begin{itemize}
	\item Vertraulichkeit
	\item Integrität
	\item Verfügbarkeit
\end{itemize}

\end{document}